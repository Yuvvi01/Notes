\documentclass[british]{article}
\usepackage{fontspec}
\setmonofont{JetBrains Mono Medium}
\usepackage[a4paper]{geometry}
\geometry{verbose,tmargin=2cm,bmargin=2cm,lmargin=2cm,rmargin=2cm}
\usepackage{fancyhdr}
\pagestyle{fancy}
\usepackage{setspace}
\usepackage{microtype}
\onehalfspacing
\usepackage[unicode=true,pdfusetitle,
 bookmarks=true,bookmarksnumbered=true,bookmarksopen=false,
 breaklinks=false,pdfborder={0 0 0},pdfborderstyle={},backref=section,colorlinks=false]
 {hyperref}
\makeatletter
\providecommand\textquotedblplain{%
  \bgroup\addfontfeatures{RawFeature=-tlig}\char34\egroup}
\fancyhead[RO]{\textsl{TABLE OF CONTENTS}}
\fancyhead{}
\fancyhead[RE,LO]{\leftmark}
\AtBeginDocument{
  \def\labelitemi{\(\star\)}
}

\makeatother

\usepackage{polyglossia}
\setdefaultlanguage[variant=british]{english}
\begin{document}
\title{\textbf{MARKDOWN NOTES}}
\author{AGNI DATTA}

\maketitle
{\small{}\tableofcontents{}}{\small\par}

\pagebreak{}

\section{Markdown:}
\begin{itemize}
\item Lightweight Markup language with a plain text formatting syntax.
\item No tags. 
\item Can be easily converted into HTML/XHTML/\LaTeX\: and other formats.
\item It is extremely readable, understandable and easily edited.
\item WYSISYG type format with a few formatting options.
\item Highly portable.
\end{itemize}

\section{Uses of Markdown:}
\begin{itemize}
\item Technical Documentation.
\item Posts in websites.
\item Used in static sites generators.
\end{itemize}

\section{Basic Formatting:}

\subsection{Headings:}
\begin{verbatim}
>> # - Heading 1
>> ## - Heading 2
>> ### - Heading 3
>> #### - Heading 4
>> #####.....N - Heading N
\end{verbatim}

\subsection{Italics:}
\begin{verbatim}
>> *Text Inside Will Be Italics*
>> _Text Inside Will Be Italics_
\end{verbatim}

\subsection{Strong Text:}
\begin{verbatim}
>> **Text Inside Will Be Strong**
>> __Text Inside Will Be Strong__
\end{verbatim}

\subsection{Strike-trough Text:}
\begin{verbatim}
>> ~~Text Inside Will Be Stroked Out~~
\end{verbatim}

\subsection{Horizontal Rule:}
\begin{verbatim}
>> - - -
>> ____
\end{verbatim}

\subsection{Escape Sequence for Special Character:}
\begin{verbatim}
>> /Special Sequence/
\end{verbatim}

\subsection{Block Quote Text:}
\begin{verbatim}
>> >This Is A Quote
\end{verbatim}

\subsection{Link Text:}
\begin{verbatim}
>> [Name Of The Link]{Actual Link "Title Of The Link"}
\end{verbatim}

\subsection{Unordered List:}
\begin{verbatim}
>> * Point 1
>> * Point 2
         * Nested Point 1
\end{verbatim}

\subsection{Ordered List:}
\begin{verbatim}
>> 1. Item 1
>> 2. Item 2
>> 3. Item 3
>> 4. Item 4
\end{verbatim}

\subsection{Inline Code Block:}
\begin{verbatim}
>> '<p> This Is A Paragraph <p>'
\end{verbatim}

\subsection{Image:}
\begin{verbatim}
>> ![Image Name] (Location Of The Image)
\end{verbatim}

\subsection{Code Blocks:}
\begin{verbatim}
>>  ''' (Language Syntax)
	Code 
	....................
	'''
\end{verbatim}

\subsection{Tables:}
\begin{verbatim}
>>  | Column 1 | Column 2 |
>>  | -------- | -------- |
>>  | Data     | Data     |
>>  | Data     | Data     |
\end{verbatim}

\subsection{Task Lists:}
\begin{verbatim}
>> * [X] Task 1
>> * [X] Task 2
>> * [ ] Task 3
\end{verbatim}

\end{document}
